\documentclass[12pt]{beamer}
%\usepackage{mystyle}

% Since <T>LAPACK does not properly render in LaTeX without putting < in math mode,
%   we define the following command to make referencing it more convenient
\newcommand{\tlapack}{$<$T$>$LAPACK}

\author{Johnathan Rhyne}
\title{Developing with \tlapack}

\begin{document}
    \begin{frame}
        \maketitle
    \end{frame}
    \begin{frame}
        \tableofcontents
    \end{frame}
    \section{What is \tlapack?}
    \begin{frame}
        \frametitle{What is \tlapack?}
        2-3 slides discussing that \tlapack is by discussing what's in the README of the 
        repo focusing on the templating paradigm and its beenfits
    \end{frame}
    \section{Developing for \tlapack}
    \begin{frame}
        \frametitle{Developing for \tlapack}
        Some slides that will discuss how code is written for \tlapack (header files, 
        compilation of it, and calling it)
    \end{frame}
    \section{My project within \tlapack}
    \begin{frame}
        \frametitle{My project within \tlapack}
        Discuss how I wrote the code for \url{https://github.com/jprhyne/tlapack/tree/hqr}
    \end{frame}
    \section{Examples of calling code}
    \begin{frame}
        Explore some of the testing files in the repo showing how we construct calls (discuss some of the nuances 
        especially wrt the templating) 
    \end{frame}
    \section{Timing of my project}
    \begin{frame}
        \frametitle{Timing of my project}
        Present some timing (performance) of things in email thread to show how we might use this as well as 
        some motivation about why we care
    \end{frame}
    \section{Differing precisions}
    \begin{frame}
        \frametitle{Differing precisions}
        At least reference the existence of using differing precisions even if we don't have time to implement it
        and show results.
    \end{frame}
\end{document}
